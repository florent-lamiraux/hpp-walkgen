\documentclass {article}

\usepackage {amsmath}
\usepackage{amssymb}

\usepackage {amsfonts} % pour les lettres maths creuses \mathbb

\newcommand\real{\mathbb{R}}
\newcommand\x{\mathbf{x}}
\newcommand\dx{\mathbf{\dot{x}}}
\newcommand\ddx{\mathbf{\ddot{x}}}
\newcommand\y{\mathbf{y}}
\newcommand\dy{\dot{\mathbf{y}}}
\newcommand\step{\mathbf{s}}
\newcommand\com{\mathbf{c}}
\newcommand\z{z}
\newcommand\zmp{\mathbf{zmp}}
\newcommand\zmpref{\mathbf{zmp}_{ref}}
\newcommand\zmprefx{\mathbf{zmp}_{ref\,0}}
\newcommand\zmprefy{\mathbf{zmp}_{ref\,1}}
\newcommand\zmprefi{\mathbf{zmp}_{ref\,i}}

\title {Cubic B-spline based walk generator}
\author {Florent Lamiraux}

\begin{document}
\maketitle

\section {Cubic B splines}

Let
\begin{itemize}
\item $m$ be an integer bigger than 7,
\item $t_0 \leq t_1 \leq \cdots \leq t_{m-1}$ an increasing sequence
  of real values,
\item $\x_{i} \in \real^2, 0\leq i \leq m-5$ control points in the
  plane.
\end{itemize}
We define the curve $\x$ from interval $[t_3, t_{m-4}]$ in $\real^2$ as
\begin{equation}\label{eq:com-1}
  \x = \sum_{i=0}^{m-5}b_{i,3}\x_i
\end{equation}\label{eq:b-spline}
where $b_{i,3}$ are the basis function of cubic B splines:
\begin{eqnarray*}
b_{i,3} &=& (B_{i, i} \mathbb{I}_{[t_i, t_{i+1})} + B_{i,i+1} \mathbb{I}_{[t_{i+1}, t_{i+2})} +B_{i,i+2} \mathbb{I}_{[t_{i+2}, t_{i+3})}+B_{i,i+3} \mathbb{I}_{[t_{i+3}, t_{i+4})})\\
&&i=0,\ldots, m-5
\end{eqnarray*}
with
\begin{eqnarray*}
B_{i, i} (t) &=& \frac{(t - t_i)^3}{(t_{i+3} - t_i)(t_{i+2} - t_i)(t_{i+1} - t_i)} \\
B_{i,i+1} (t) &=& \frac{(t - t_i)^2(t_{i+2} - t)}{(t_{i+3} - t_i)(t_{i+2} - t_{i+1})(t_{i+2} - t_i)}+\frac{(t - t_i)(t_{i+3} - t)(t - t_{i+1})}{(t_{i+3} - t_i)(t_{i+3} - t_{i+1})(t_{i+2} - t_{i+1})}\\
&&+\frac{(t_{i+4} - t)(t - t_{i+1})^2}{(t_{i+4} - t_{i+1})(t_{i+3} - t_{i+1})(t_{i+2} - t_{i+1})} \\
B_{i,i+2} (t) &=& \frac{(t - t_i)(t_{i+3} - t)^2}{(t_{i+3} - t_i)(t_{i+3} - t_{i+1})(t_{i+3} - t_{i+2})}+\frac{(t_{i+4} - t)(t - t_{i+1})(t_{i+3} - t)}{(t_{i+4} - t_{i+1})(t_{i+3} - t_{i+1})(t_{i+3} - t_{i+2})}\\
&&+\frac{(t_{i+4} - t)^2(t - t_{i+2})}{(t_{i+4} - t_{i+1})(t_{i+4} - t_{i+2})(t_{i+3} - t_{i+2})}\\
B_{i,i+3} (t) &=& \frac{(t_{i+4} - t)^3}{(t_{i+4} - t_{i+1})(t_{i+4} - t_{i+2})(t_{i+4} - t_{i+3})}
\end{eqnarray*}
and for any interval $I$ of $\real$, $\mathbb{I}_{I}$ is the function equal to
1 over $I$ and to $0$ outside $I$.

\section {Walk generator}

\subsection {Input}

The input of the walk generator is a sequence of time-stamped steps defined
as follows:
\begin{enumerate}
\item $p$ is a positive integer not smaller than 3,
\item $\tau_0, \tau_1, \cdots, \tau_{2p-3}$ is an increasing sequence of real values,
\item $\step_0, \step_1, \cdots, \step_{p-1}$ is a sequence of $p$ elements of
  $\real^2$ representing the successive positions of the foot centers.
\item $\com_{init}\in\real^2$ is the initial position of the center of mass at
  time $\tau_0$,
\item $\com_{final}\in\real^2$ is the final position of the center of mass at time
  $\tau_{2p-3}$.
\end{enumerate}

\subsection {Reference trajectory of the center of pressure}

We define a reference trajectory of the center of pressure called
$\zmpref$ as a continuous piecewise affine curve as follows:
\begin{eqnarray*}
\zmpref (\tau_0) &=& \com_{init} \\
\zmpref (\tau_1) &=& \step_1 \\
\zmpref (\tau_2) &=& \step_1 \\
\zmpref (\tau_3) &=& \step_2 \\
\vdots & \vdots & \vdots \\
\zmpref (\tau_{2p-5}) &=& \step_{p-2} \\
\zmpref (\tau_{2p-4}) &=& \step_{p-2} \\
\zmpref (\tau_{2p-3}) &=& \com_{final}
\end{eqnarray*}
such that $\zmpref$ restricted to each interval $[\tau_i,\tau_{i+1}]$ with
$i\in\{0,1,\cdots,2p-4\}$ is affine.

\subsection {Trajectory of the center of mass}

We restrict the trajectory of the center of mass to be a cubic-spline defined by
Equation~(\ref{eq:b-spline}). We want
\begin {itemize}
\item the whole center of mass trajectory to be defined on interval $[\tau_0, \tau_{2p-3}]$, and
\item $l\geq 1$ knots on each interval $[\tau_i,\tau_{i+1})$, $i\in\{0,\cdots,2p-3\}$.
\end{itemize}
We get the following relations
between the various parameters:
\begin {eqnarray*}
m &=& (2p-3)l+7 \\
t_3 &=& \tau_0 \\
t_{m-4} &=& \tau_{2p-3} \\
\end {eqnarray*}
We set
\centerline {
  \parbox {.49\linewidth} {
    \begin {eqnarray*}
      t_0 &=& \tau_0 - 3 \\
      t_1 &=& \tau_0 - 2 \\
      t_2 &=& \tau_0 - 1 \\
    \end {eqnarray*}
  }
  \parbox {.49\linewidth} {
    \begin {eqnarray*}
      t_{m-3} &=& \tau_{2p-3} + 1 \\
      t_{m-2} &=& \tau_{2p-3} + 2 \\
      t_{m-1} &=& \tau_{2p-3} + 3 \\
    \end {eqnarray*}
  }
}
and
\begin{eqnarray*}
t_{3 + k\,l + j} &=& \frac{l-j}{l}\, \tau_{k} + \frac{j}{l}\, \tau_{k+1}\ \ \mbox{for } j\in\{0,\cdots,\l-1\}, \ k\in\{0,\cdots,2p-4\}
\end{eqnarray*}

\subsection {Trajectory of the center of pressure}

Let $g$ be the gravity constant, and $z$ the constant height of the center
of mass. By denoting $\omega = \sqrt {g/z}$, we get the simplified formula for
the center of pressure of the robot:
$$
\zmp = \x - \frac{1}{\omega^2} \ddx
$$
By setting
$$
\z_{i,3} \triangleq b_{i,3} - \frac{1}{\omega^2} \ddot{b}_{i,3}
$$
we get an expression of the center of pressure with respect to the control
points of the cubic B spline:
\begin{equation}\label{eq:zmp-spline}
\zmp = \sum_{i=0}^{m-5} \z_{i,3} \x_{i}= \sum_{i=0}^{2p-1} \z_{i,3} \x_{i}
\end{equation}

\subsection {Optimal control problem}

We denote by $X=(\x_0, \cdots, \x_{m-5})$ the vector of control points.
We wish to find the cubic B spline trajectory that minimizes the
following cost function:
\begin{equation}\label{eq:qudratic-cost}
C(X) \triangleq \frac{1}{2}\int_{\tau_0}^{\tau_{2p-3}} \|\zmp (t) - \zmpref (t)\|^2 dt
\end{equation}
Let us expand this formula using~(\ref{eq:zmp-spline}):
\begin{eqnarray*}
C(X) &=& \frac{1}{2}\int_{\tau_0}^{\tau_{2p-3}} (\sum_{i=0}^{2p-4} \z_{i,3} \x_{i} - \zmpref (t))^T(\sum_{i=0}^{2p-4} \z_{i,3} \x_{i} - \zmpref (t)) dt\\
&=& \frac{1}{2}\sum_{i=0}^{m-5}\sum_{j=0}^{m-5}\int_{\tau_0}^{\tau_{2p-3}} \z_{i,3} \z_{j,3} (t) dt\  \x_i^T\x_j - \sum_{i=0}^{m-5}\int_{\tau_0}^{\tau_{2p-3}}\z_{i,3}\zmpref^T(t)dt\ \x_i\\
&&+ \int_{\tau_0}^{\tau_{2p-3}} \zmpref^T\zmpref (t)dt
\end{eqnarray*}
The cost function can be rewritten as
\begin{eqnarray*}
C(X) &=& \frac{1}{2}X^THX - b^T X + C_0
\end{eqnarray*}
with
\begin{eqnarray*}
H &=& \left(\begin{array}{c}
\int_{\tau_0}^{\tau_{2p-3}} \z_{i,3} \z_{j,3}(t)dt\ I_2 \end{array}\right)_{i,j=0,\cdots,m-5} \\
b &=& \left(\begin{array}{c}\int_{\tau_0}^{\tau_{2p-3}}\z_{i,3}\zmpref(t)dt\end{array}\right)_{i=0,\cdots,m-5}\\
C_0 &=&\int_{\tau_0}^{\tau_{2p-3}} \zmpref^T\zmpref (t)dt
\end{eqnarray*}
$C(X)$ is the sum of two terms that respectively depend on the $x$ and $y$
coordinates of the control points. Let us denote $\x_{i,0}$, $\x_{i,1}$ the abscissa and the ordinate of $\x_i$, $\zmprefx$, $\zmprefy$ the abscissa and ordinate of $\zmpref$, and let us define
\begin{eqnarray*}
  X_0 &=& (\x_{0,0},\cdots,\x_{m-5,0}) \\
  X_1 &=& (\x_{0,1},\cdots,\x_{m-5,1}) \\
  b_0 &=& \left(\begin{array}{c}\int_{\tau_0}^{\tau_{2p-3}}\z_{i,3}\ \zmprefx(t)dt\end{array}\right)_{i=0,\cdots,m-5}\\
  b_1 &=& \left(\begin{array}{c}\int_{\tau_0}^{\tau_{2p-3}}\z_{i,3}\ \zmprefy(t)dt\end{array}\right)_{i=0,\cdots,m-5}\\
  H_0 = H_1 &=& \left(\begin{array}{c}
    \int_{\tau_0}^{\tau_{2p-3}} \z_{i,3} \z_{j,3}(t)dt \end{array}\right)_{i,j=0,\cdots,m-5} \\
\end{eqnarray*}
Then,
\begin{eqnarray*}
C(X) &=& \frac{1}{2}X_0^TH_0X_0 - b_0^T X_0 + \frac{1}{2}X_1^TH_1X_1 - b_1^T X_1 + C_0
\end{eqnarray*}
The problem can therefore be decomposed into two decoupled sub-problems, one in $X_0$ and the other one in $X_1$.

\subsubsection {Linear constraints}

Boundary conditions can be added as a constraint on the value of the
trajectory of the center of mass and its derivative at a given
parameter -- usually at the beginning or at the end of the definition
interval. Each of these constraints is defined by a tuple
$(t,\y,\dy)\in[\tau_0,\tau_{2p-3}]\times\real^2\times\real^2$ and is
  linear in the vector of control points.
\begin{eqnarray}\label{eq:constraint}
\sum_{i=0}^{m-5}b_{i,3}(t)\,\x_i &=&\y \\
\sum_{i=0}^{m-5}\dot{b}_{i,3}(t)\,\x_i &=& \dy
\end{eqnarray}
These constraints can be translated to each sub-problem as follows:
\begin{eqnarray*}
  A_0\,X_0 &=& c_0 \\
  A_1\,X_1 &=& c_1 \\
\end{eqnarray*}
with
\begin{eqnarray*}
  A_0 = A_1 &\triangleq& \left(\begin{array}{cccc}b_{0,3}(t)&b_{1,3}(t)&\cdots&b_{m-5,3}(t)\\
    \dot{b}_{0,3}(t)&\dot{b}_{1,3}(t)&\cdots&\dot{b}_{m-5,3}(t)\end{array}\right) \\
  c_0 &=& \left(\begin{array}{c}\y_0\\\dy_0\end{array}\right) \\
  c_1 &=& \left(\begin{array}{c}\y_1\\\dy_1\end{array}\right) \\
\end{eqnarray*}

\subsubsection {Resolution of the quadratic program}

The constrained problem can be expressed as follows for $i\in{0,1}$:
$$
\min_{X_i} \frac{1}{2}X_i^TH_iX_i - b_i^TX_i \ \ \mbox{such that}\ \ A_i\,X=c_i
$$
using the singular value decomposition of $A_i$
$$
A_i = \left(\begin{array}{cc}U_1 & U_0\end{array}\right) \Sigma
\left(\begin{array}{cc}V_1 & V_0\end{array}\right)^T
$$
we get a parameterization of the affine sub-space defined by the constraint:
$$
X_i = X_{i\,0} + V_0\mathbf{u}\ \ \ \mathbf{u}\in\real^{m-4-rank(A_i)}
$$
where $X_{i\,0}=A_i^{+}c_i$.
Solving the constrained QP consists in finding $\mathbf{u}$ that minimizes
\begin{eqnarray*}
&&\frac{1}{2}( X_{i\,0} + V_0\mathbf{u})^TH_i( X_{i\,0} + V_0\mathbf{u}) - b_i^T( X_{i\,0} + V_0\mathbf{u})\\
&=& \frac{1}{2}\mathbf{u}^TV_0^TH_iV_0\mathbf{u} + X_{i\,0}^T H_i V_0\mathbf{u} - b_i^TV_0\mathbf{u} + Cste \\
&=& \frac{1}{2}\mathbf{u}^TV_0^TH_iV_0\mathbf{u} + (V_0^T H_i X_{i\,0}  - V_0^Tb_i)^T\mathbf{u} + Cste \\
\end{eqnarray*}
The value of $\mathbf{u}$ that minimizes the above expression is given by
$$
\mathbf{u_i}^{*} = (V_0^TH_iV_0)^{-1}(V_0^Tb_i  - V_0^T H_i X_{i\,0})
$$
\subsection {Computation of the coefficients}

\begin{eqnarray*}
z_{i,3} &=& (Z_{i, i} \mathbb{I}_{[t_i, t_{i+1})} + Z_{i,i+1} \mathbb{I}_{[t_{i+1}, t_{i+2})} +Z_{i,i+2} \mathbb{I}_{[t_{i+2}, t_{i+3})}+Z_{i,i+3} \mathbb{I}_{[t_{i+3}, t_{i+4})})\\
&&i=0,\ldots, m-5
\end{eqnarray*}
with
\begin{eqnarray*}
Z_{i, i} (t) &=& B_{i,i} - \frac{1}{\omega^2} \ddot{B}_{i,i} \\
Z_{i,i+1} (t) &=& B_{i,i+1} - \frac{1}{\omega^2} \ddot{B}_{i,i+1} \\
Z_{i,i+2} (t) &=& B_{i,i+2} - \frac{1}{\omega^2} \ddot{B}_{i,i+2} \\
Z_{i,i+3} (t) &=& B_{i,i+3} - \frac{1}{\omega^2} \ddot{B}_{i,i+3} \\
\end{eqnarray*}
Matrix $H_0$ is symmetric. For any integer $i$ such that $0\leq i \leq m-5$, and any non-negative integer $k$, such that $k \leq 3$ and $i+k \leq m-5$,
The coefficient $(i,i+k)$ of matrix $H_0$, with $k\in\{0,1,2,3\}$ is equal to
\begin{eqnarray*}
H_{0\ i,i+k} &=& \int_{\tau_0}^{\tau_{2p-3}} \z_{i,3} \z_{i+k,3}(t)dt\\
%% &=& \int_{\tau_0}^{\tau_{2p-3}} (Z_{i, i} \mathbb{I}_{[t_i, t_{i+1})} + Z_{i,i+1} \mathbb{I}_{[t_{i+1}, t_{i+2})} +Z_{i,i+2} \mathbb{I}_{[t_{i+2}, t_{i+3})}+Z_{i,i+3} \mathbb{I}_{[t_{i+3}, t_{i+4})})\\
%% && \hspace*{1cm}(Z_{i+k, i+k} \mathbb{I}_{[t_i+k, t_{i+k+1})} + Z_{i+k,i+k+1} \mathbb{I}_{[t_{i+k+1}, t_{i+k+2})} +\\
%% && \hspace*{1cm}Z_{i+k,i+k+2} \mathbb{I}_{[t_{i+k+2}, t_{i+k+3})}+Z_{i+k,i+k+3} \mathbb{I}_{[t_{i+k+3}, t_{i+k+4})}) dt\\
&=& \sum_{j=0}^{3-k} \int_{t_{i+k+j}}^{t_{i+k+j+1}}Z_{i,i+k+j} Z_{i+k,i+k+j} (t) dt
\end{eqnarray*}
The coefficients of vector $b_0$ are equal to:
\begin{eqnarray*}
  b_{0\,i} &=& \int_{t_3}^{t_{m-4}}(Z_{i, i} \mathbb{I}_{[t_i, t_{i+1})} + Z_{i,i+1} \mathbb{I}_{[t_{i+1}, t_{i+2})} +Z_{i,i+2} \mathbb{I}_{[t_{i+2}, t_{i+3})}+Z_{i,i+3} \mathbb{I}_{[t_{i+3}, t_{i+4})})\ \zmprefi(t)dt \\
  %% b_{0\,0} &=& \int_{t_3}^{t_4} Z_{0,0+3}\ \zmprefi(t)dt \\
  %% b_{0\,1} &=& \int_{t_3}^{t_4} Z_{1,1+2}\ \zmprefi(t)dt + \int_{t_4}^{t_5} Z_{1,1+3}\ \zmprefi(t)dt \\
  %% b_{0\,2} &=& \int_{t_3}^{t_4} Z_{2,2+1}\ \zmprefi(t)dt + \int_{t_4}^{t_5} Z_{2,2+2}\ \zmprefi(t)dt + \int_{t_5}^{t_6} Z_{2,2+3}\ \zmprefi(t)dt \\
  %% b_{0\,i} &=& \int_{t_{i+1}}^{t_{i+2}} Z_{i,i+0} \zmprefi(t)dt + \int_{t_{i+2}}^{t_{i+3}} Z_{i,i+1} \zmprefi(t)dt\\
  %% && + \int_{t_{i+3}}^{t_{i+4}} Z_{i,i+2} \zmprefi(t)dt + \int_{t_{i+4}}^{t_{i+5}} Z_{i,i+3} \zmprefi(t)dt\\
  %% &=& \sum_{j=0}^{3} \int_{t_{i+1+j}}^{t_{i+2+j}} Z_{i,i+j} \zmprefi(t)\,dt\\
  &=& \sum_{j=\max(0,3-i)}^{\min(3,m-5-i)} \int_{t_{i+j}}^{t_{i+j+1}} Z_{i,i+j} \zmprefi(t)\,dt
\end{eqnarray*}

\subsubsection* {Special cases for boundary conditions}
If $t = \tau_0 = t_3$,
\begin{eqnarray*}
  A_0 = A_1 &=& \left(\begin{array}{cccccc}B_{0,0+3}(t_3)&B_{1,1+2}(t_3)&B_{2,2+1}(t_3)&0&\cdots&0\\
    \dot{B}_{0,0+3}(t_3)&\dot{B}_{1,1+2}(t_3)&\dot{B}_{2,2+1}(t_3)&0&\cdots&0\end{array}\right) \\
\end{eqnarray*}
If $t = \tau_{2p-3} = t_{m-4}$
\begin{eqnarray*}
  A_0 = A_1 &=& \left(\begin{array}{cccccc}0&\cdots&0&B_{m-7,m-4}(t_{m-4})&B_{m-6,m-4}(t_{m-4})&B_{m-5,m-4}(t_{m-4})\\
    0&\cdots&0&\dot{B}_{m-7,m-4}(t_{m-4})&\dot{B}_{m-6,m-4}(t_{m-4})&\dot{B}_{m-5,m-4}(t_{m-4})\\\end{array}\right) \\
\end{eqnarray*}
\end{document}
